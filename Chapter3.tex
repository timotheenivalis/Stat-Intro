\documentclass{beamer}

\usepackage{alltt}%
\usetheme{Boadilla}
\usecolortheme{seahorse}

%\usepackage{listings}
\makeatletter
\def\maxwidth{ %
  \ifdim\Gin@nat@width>\linewidth
    \linewidth
  \else
    \Gin@nat@width
  \fi
}
\makeatother

\definecolor{fgcolor}{rgb}{0.345, 0.345, 0.345}
\newcommand{\hlnum}[1]{\textcolor[rgb]{0.686,0.059,0.569}{#1}}%
\newcommand{\hlstr}[1]{\textcolor[rgb]{0.192,0.494,0.8}{#1}}%
\newcommand{\hlcom}[1]{\textcolor[rgb]{0.678,0.584,0.686}{\textit{#1}}}%
\newcommand{\hlopt}[1]{\textcolor[rgb]{0,0,0}{#1}}%
\newcommand{\hlstd}[1]{\textcolor[rgb]{0.345,0.345,0.345}{#1}}%
\newcommand{\hlkwa}[1]{\textcolor[rgb]{0.161,0.373,0.58}{\textbf{#1}}}%
\newcommand{\hlkwb}[1]{\textcolor[rgb]{0.69,0.353,0.396}{#1}}%
\newcommand{\hlkwc}[1]{\textcolor[rgb]{0.333,0.667,0.333}{#1}}%
\newcommand{\hlkwd}[1]{\textcolor[rgb]{0.737,0.353,0.396}{\textbf{#1}}}%
\let\hlipl\hlkwb

\usepackage{framed}
\makeatletter
\newenvironment{kframe}{%
 \def\at@end@of@kframe{}%
 \ifinner\ifhmode%
  \def\at@end@of@kframe{\end{minipage}}%
  \begin{minipage}{\columnwidth}%
 \fi\fi%
 \def\FrameCommand##1{\hskip\@totalleftmargin \hskip-\fboxsep
 \colorbox{shadecolor}{##1}\hskip-\fboxsep
     % There is no \\@totalrightmargin, so:
     \hskip-\linewidth \hskip-\@totalleftmargin \hskip\columnwidth}%
 \MakeFramed {\advance\hsize-\width
   \@totalleftmargin\z@ \linewidth\hsize
   \@setminipage}}%
 {\par\unskip\endMakeFramed%
 \at@end@of@kframe}
\makeatother

\definecolor{shadecolor}{rgb}{.97, .97, .97}
\definecolor{messagecolor}{rgb}{0, 0, 0}
\definecolor{warningcolor}{rgb}{1, 0, 1}
\definecolor{errorcolor}{rgb}{1, 0, 0}
\newenvironment{knitrout}{}{} % an empty environment to be redefined in TeX


\usepackage[utf8]{inputenc}
\usepackage{default}

\usepackage{xcolor}%for color mixing

\usepackage{amsmath}%
\usepackage{amsfonts}%
\usepackage{amssymb}%
\usepackage{graphicx}

\usepackage{tikz}
\usepackage{multirow}
\usepackage{booktabs}

\setbeamertemplate{itemize/enumerate body begin}{\small}

%%%%%%%%%%%%%%%%%%%%%%%%%%%%%%%%%%%%%%%%%%%%%%%%%%%%%%%%%%%%%%%%%%%%%%%%%%%%%%%%%%

\title{Statistical Modelling: Understanding Mean Structure}
\subtitle{Chapter 3}
\author{Terry Neeman and Timoth\'ee Bonnet}
\date{\today}

\begin{document}

%\lstset{language=R}%code

\AtBeginSection[]
{
  \begin{frame}<beamer>
    \frametitle{}
    \tableofcontents[currentsection,sectionstyle=show/show,subsectionstyle=show/shaded/hide]% down vote\tableofcontents[currentsection,currentsubsection,hideothersubsections,sectionstyle=show/hide,subsectionstyle=show/shaded/hide] 
  \end{frame}
}


\begin{frame}{}
\maketitle

\end{frame}
%%%%%%%%%%%%%%%%%%%%%%%


\begin{frame}{Key components of a statistical model of an experiment}
\begin{itemize}
  \item Outcome measure
  \begin{itemize}
   \item Response variable
   \item Measure of interest
  \end{itemize}
  \item Experimental factors 
  \begin{itemize}
   \item Conditions that can be manipulated 
   \item Conditions of interest (e.g. genotype, gender) 
   \item Main questions: do the conditions impact upon the outcome measure?
  \end{itemize}
  \item Blocking factors
  \begin{itemize}
   \item Conditions (not of interest) that may impact upon the outcome measure
   \item Sources of variation in the experiment that need to be controlled for
   \item Clustering of experimental units
  \end{itemize}
\end{itemize}

\vspace{0.2cm}
ALWAYS BEGIN WITH A RESEARCH QUESTION

\end{frame}
%%%%%%%%%%%%%%%%%%%%%%%

\begin{frame}{Example 1: Can drought tolerance in Arabidopsis be improved through genetic modification?}

\begin{block}{Context}
  Outcome measure: Leaf water retention LWR (\%)\\
  Experimental factors:
    \begin{itemize}
     \item Gene A, genotypes \texttt{(AA/aa)}
     \item Gene B, genotypes \texttt{(BB/bb)}
    \end{itemize}
\end{block}
How many parameters?
        
\pause

\begin{center}
\begin{tabular}{|l | l | l | l | }
\toprule
  \multicolumn{2}{|l|}{4 treatments} & \multicolumn{2}{l|}{Gene A}\\
  \cmidrule(lr){3-4}
  \multicolumn{2}{|l|}{}  & AA & aa\\
 	    \midrule
      Gene B & BB & $C$ & $C+A$\\
      \cmidrule(lr){3-4}
 	    & bb & $C+B$ & $C+A+B+D$\\
	    \bottomrule
  \end{tabular}
\end{center}
 
\end{frame}
%%%%%%%%%%%%%%

\begin{frame}{Two different models}

\textbf{Additive model - 3 parameters}
\begin{center}
\begin{tabular}{|l | l | l | l | }
\toprule
  \multicolumn{2}{|l|}{4 treatments} & \multicolumn{2}{l|}{Gene A}\\
  \cmidrule(lr){3-4}
  \multicolumn{2}{|l|}{}  & AA & aa\\
 	    \midrule
      Gene B & BB & $C$ & $C+A$\\
      \cmidrule(lr){3-4}
 	    & bb & $C+B$ & $C+A+B$\\
	    \bottomrule
  \end{tabular}
\end{center}
 

\textbf{Full factorial model / Interactive model  - 4 parameters }
\begin{center}
\begin{tabular}{|l | l | l | l | }
\toprule
  \multicolumn{2}{|l|}{4 treatments} & \multicolumn{2}{l|}{Gene A}\\
  \cmidrule(lr){3-4}
  \multicolumn{2}{|l|}{}  & AA & aa\\
 	    \midrule
      Gene B & BB & $C$ & $C+A$\\
      \cmidrule(lr){3-4}
 	    & bb & $C+B$ & $C+A+B+D$\\
	    \bottomrule
  \end{tabular}
\end{center}
 

 \textbf{\emph{What is different? What does the additive model assume?}}
\end{frame}
%%%%%%%%%%%%

\begin{frame}{Which model to use?}
 \textbf{Additive model - 3 parameters}
\begin{center}
\begin{tabular}{|l | l | l | l | }
\toprule
  \multicolumn{2}{|l|}{4 treatments} & \multicolumn{2}{l|}{Gene A}\\
  \cmidrule(lr){3-4}
  \multicolumn{2}{|l|}{}  & AA & aa\\
 	    \midrule
      Gene B & BB & $C$ & $C+A$\\
      \cmidrule(lr){3-4}
 	    & bb & $C+B$ & $C+A+B$\\
	    \bottomrule
  \end{tabular}
\end{center}
 

\textbf{Full factorial model / Interactive model - 4 parameters }
\begin{center}
\begin{tabular}{|l | l | l | l | }
\toprule
  \multicolumn{2}{|l|}{4 treatments} & \multicolumn{2}{l|}{Gene A}\\
  \cmidrule(lr){3-4}
  \multicolumn{2}{|l|}{}  & AA & aa\\
 	    \midrule
      Gene B & BB & $C$ & $C+A$\\
      \cmidrule(lr){3-4}
 	    & bb & $C+B$ & $C+A+B+D$\\
	    \bottomrule
  \end{tabular}
\end{center}
\end{frame}
%%%%%%%%%%%

\begin{frame}{Analysis in R}
 
 \begin{enumerate}
  \item Import data ''Prac3mockLWR.csv''
  \item Visualize data
  \item Model data
  \item Assess model assumptions
 \end{enumerate}

\end{frame}
%%%%%%%%%%%

\begin{frame}[fragile]{Analysis in R}
 
 1. Import data ''Prac3mockLWR.csv''
 
 \begin{knitrout}
\definecolor{shadecolor}{rgb}{0.969, 0.969, 0.969}\color{fgcolor}\begin{kframe}
\begin{verbatim}
  LWR <- read.csv(“Prac3mockLWR.csv")
 \end{verbatim}
\end{kframe}
\end{knitrout}
 
\end{frame}
%%%%%%%%%%%

\begin{frame}[fragile]{Analysis in R}
 
2. Visualise the data

 \begin{knitrout}
\definecolor{shadecolor}{rgb}{0.969, 0.969, 0.969}\color{fgcolor}\begin{kframe}
\begin{verbatim}

ggplot(LWR, aes(GeneB,LWR,colour=GeneA)) +
  geom_boxplot() + geom_point()
 \end{verbatim}
\end{kframe}
\end{knitrout}
 
 Full factorial or additive?
 
\end{frame}
%%%%%%%%%%%



\begin{frame}[fragile]{Analysis in R}
 
3. Model data

 \begin{knitrout}
\definecolor{shadecolor}{rgb}{0.969, 0.969, 0.969}\color{fgcolor}\begin{kframe}
\begin{verbatim}
lmadditive <- lm(LWR ~ GeneA + GeneB, data = LWR)
summary(lmadditive)
anova(lmadditive)
 \end{verbatim}
\end{kframe}
\end{knitrout}

 \begin{knitrout}
\definecolor{shadecolor}{rgb}{0.969, 0.969, 0.969}\color{fgcolor}\begin{kframe}
\begin{verbatim}
lminteraction <- lm(LWR ~ GeneA * GeneB, data = LWR)
summary(lminteraction)
anova(lminteraction)
emmeans(lminteraction, pairwise ~ GeneA|GeneB)
emmeans(lminteraction, pairwise ~ GeneB|GeneA)
 \end{verbatim}
\end{kframe}
\end{knitrout}

What are the estimates for $A, B, C, D$ under each models?
\end{frame}
%%%%%%%%%%%



\begin{frame}[fragile]{Analysis in R}
 
4. Model assumptions

 \begin{knitrout}
\definecolor{shadecolor}{rgb}{0.969, 0.969, 0.969}\color{fgcolor}\begin{kframe}
\begin{verbatim}
plot(lminteraction)
 \end{verbatim}
\end{kframe}
\end{knitrout}

\end{frame}
%%%%%%%%%%%

\end{document}