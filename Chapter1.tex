\documentclass{beamer}

\usepackage{alltt}%
\usetheme{Boadilla}
\usecolortheme{seahorse}

\usepackage[utf8]{inputenc}
\usepackage{default}

\usepackage{xcolor}%for color mixing

\usepackage{amsmath}%
\usepackage{amsfonts}%
\usepackage{amssymb}%
\usepackage{graphicx}

%%%%%%%%%%%%%%%%%%%%%%%%%%%%%%%%%%%%%%%%%%%%%%%%%%%%%%%%%%%%%%%%%%%%%%%%%%%%%%%%%%

\title{Statisitcal Thinking in Biology Research}
\author{Terry Neeman and Timothee Bonnet}
\date{\today}

\begin{document}

\begin{frame}{}
\maketitle

\end{frame}
%%%%%%%%%%%%%%%%%%%%%%%

\begin{frame}{Acknowledgemnts and warning}

\end{frame}
%%%%%%%%%%%%%%%%%%%%%%%

\begin{frame}{Key ideas for today}

\begin{itemize}[<+->]
 \item Statistics in biology is the study of biological variation
 \item Statistical ideas about biological variation inform the design of experiments
 \item Statistical ideas about biological variation inform the analysis of experiments
 \item Statistical thinking is an essential component of scientific thinking
\end{itemize}

\end{frame}
%%%%%%%%%%%%%%%%%%%%%%%

\begin{frame}{Cautionary tales from the front}

\end{frame}
%%%%%%%%%%%%%%%%%%%%%%%

\begin{frame}{Message 1: A small p-value is not always evidence of a treatment effect}

\end{frame}
%%%%%%%%%%%%%%%%%%%%%%%

\end{document}
